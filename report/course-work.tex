\documentclass[12pt, a4paper] {ncc}

\usepackage[utf8] {inputenc}
\usepackage[T2A]{fontenc}
\usepackage[english, russian] {babel}
\let\proof\relax
\let\endproof\relax
\usepackage{amsthm}
\usepackage{amsmath}
\usepackage{amssymb}
\usepackage{listings}
\usepackage{hyperref}
\usepackage{color}
\usepackage{graphicx}
\usepackage{listings}
\usepackage{caption}
\graphicspath{ {./} }

\captionsetup{labelformat=empty,labelsep=none}

\lstset{
	basicstyle=\footnotesize,
	breakatwhitespace=false,
	breaklines=true,
	extendedchars=true,
	frame=single,
	keepspaces=true,
	keywordstyle=\bfseries,
	language=C,
	numbers=left,
	numbersep=5pt,
	numberstyle=\tiny,
	showspaces=false,
	showstringspaces=false,
	showtabs=false,
	stepnumber=1,
	stringstyle=\emph,
	tabsize=4
}
\lstset{extendedchars=\true}

\begin{document}


\textbf{Денис Ваксман, M3439}


\textbf{Курсовая работа по курсу "Введение в базы данных"}


\section{Модель сущность-связь}
\section{Физическая модель}
\section{Нормализация}
\subsection{Функциональные зависимости}
\textbf{Media:}
\begin{itemize}
\item id $\rightarrow$ filePath
\item filePath $\rightarrow$ id
\item id $\rightarrow$ type
\item id $\rightarrow$ size
\item id $\rightarrow$ time
\end{itemize}
У следующих отношений пустое множество функциональных зависимостей:
\begin{itemize}
\item UserChats
\item UserGroups
\item Friends
\end{itemize}
У следующих отношений множество функциональных зависимостей имеет вид \\
$\{id \rightarrow A ~ | ~ \forall A \in \textit{attrs}, A \ne id\}$:
\begin{itemize}
\item Users
\item Chats
\item Messages
\item Groups
\item Posts
\item UserPosts
\item UserMedia
\item GroupPosts
\item GroupMedia
\item PostMedia
\item MessageMedia
\end{itemize}
\subsection{Нормальные формы}
\textbf{Media}:
\begin{itemize}
\item 3 НФ: \\
т.к. все неключевые атрибуты непосредственно зависят от ключа
\item НФБК: \\
т.к. все ключи простые и 3НФ
\item 4НФ и 5НФ: \\
по теореме Дейта-Фейгина 1 (все ключи простые и НФБК)
\end{itemize}
TODO:
\begin{itemize}
\item UserChats
\item UserGroups
\item Friends
\end{itemize}
Оставшиеся отношения имеют множество ФЗ вида \\
$\{id \rightarrow A ~ | ~ \forall A \in \textit{attrs}, A \ne id\}$:
\begin{itemize}
\item 3 НФ: \\
т.к. все неключевые атрибуты непосредственно зависят от ключа
\item НФБК: \\
т.к. все ключи простые и 3НФ
\item 4НФ и 5НФ: \\
по теореме Дейта-Фейгина 1 (все ключи простые и НФБК)
\end{itemize}
\section{Основные SQL запросы}
\subsection{Чатсые действия}
\begin{enumerate}
\item Пользователь отправил другому пользователю сообщение
\item Пользователь опубликовал пост
\item Пользователь подписался на группу
\item Пользователь добавил в друзья другого пользователя
\item Группа опубликовала пост
\item Пользователь загрузил фотографию
\item Группа загрузила видеозапись
\item Пользователь открыл ленту (получил все посты своих групп)
\end{enumerate}
\end{document}